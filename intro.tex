\begin{flushleft}
  {\fontsize{12}{0} \bf 1. 緒論}

  \begin{abstract}
  \hspace{2em}
  PCB是電子產業的關鍵元件,其品質直接影響產品性能。根據台灣電路板協會(TPCA)最新統計,2024年上半年台商PCB全球產值達到新台幣3722億元,較2023年同期成長6\%。預計2024年全年產值將達8337億元,年增幅約8.3\%。 
  \\\hspace{2em}
  PCB製造過程會遇到的問題十分繁雜,且製程中每個步驟都可能產生瑕疵,若未及時檢測,將導致產品效能下降甚至重大損失。因此,高效且精準的瑕疵檢測是重要的研究方向之一。隨著深度學習技術的快速發展,卷積神經網絡(CNN)已成為影像分類的主要技術。從 ResNet 到 MobileNet,再到新架構的ViT(Vision Transformer)提出,多種模型在結構設計與計算效率上都各具特色,且面對不同應用場景與數據特性時,表現可能存在不同的擅長區間,因此有必要對多種模型進行系統比較,以辨識其適用性與優勢。
  \\\hspace{2em}
  本研究結合深度學習技術,針對 PCB 外觀瑕疵檢測的問題,訓練不同模型進行比較,再加上生成對抗網絡(GAN)技術的引入,可以通過數據庫增列解決資料分布不均的問題,以求進一步提高模型效能。
  \end{abstract}
\end{flushleft}