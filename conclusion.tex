\begin{flushleft}
  {\fontsize{12}{0} \bf 結論}
  \begin{abstract}
  \\\hspace{2em}
  透過實驗數據的綜合分析,得到以下結論:  
  資料擴增的影響:引入 GAN 生成的瑕疵資料後,資料多樣性有明顯提升,尤其在初期訓練階段,有助於模型 Loss 的穩定收斂。我們也觀察到GAN 擴增資料對於不同模型成效也不一,其中 ViT、MobileNetV1、GoogLeNet 等模型訓練成效都不錯,mAP分別提升約 1\% 至 2\%。  
  \\
  模型參數對表現的影響:  
  \\\hspace{2em}
  較小的Batch size有助於穩定訓練,但可能限制模型對細微瑕疵的 recall 表現。大批次大小則提升 recall 能力,但可能導致收斂速度變慢。  
  Learning rate 對模型穩定性影響十分顯著,尤其是 VGG 模型,在使用極低 Learning rate(如 2E-5)時,表現精度達到最佳,而過高的 Learning rate 則可能導致震盪以致無法收斂。  
  \\\hspace{2em}
  特殊情況的觀察:MNASNet 表現出較明顯的不穩定性,其 valid loss 波動顯著,且即使引入 GAN 資料,精度提升幅度有限。  
  \\\hspace{2em}
  整體模型表現:本實驗大中多數的模型表現較為穩定,且能適應 GAN 擴增資料的模型。而 VGG 模型則可能比較適用於極高精度的任務場景,MNASNet則需進一步調整以提升穩定性。  
  \\
  未來,本研究可在以下方向展開更深入的探索:  
  \\
  1.提升 GAN 資料的生成方法,提升其與實際資料特性的匹配程度,進一步增強模型對瑕疵的辨識能力。  
  \\
  2.受限於訓練時長,本次模型訓練的參數選擇並不能做到全面涵蓋,可能需要探索更高效的訓練策略,以加速Loss收斂與提升模型精度的目的 。  
  \\
  3.再引入更多新的深度學習模型或技術,驗證這些模型在PCD瑕疵檢測領域的適用性。 
  \end{abstract}
\end{flushleft}