\title{\fontsize{14}{0} \bf 基於深度學習技術提升PCB線路瑕疵檢測在AOI應用下的錯誤預警能力}
\author{\IEEEauthorblockN{\bf 陳旻奇、蔡沐霖、吳偉民、冷堂愷、劉晏誠、林鼎鈞}
\IEEEauthorblockA{\bf 國立高雄大學電機工程學系\\
\{a1105155|b1105129|b1115157|b1115112|a1095161|a1115162\}@mail.nuk.edu.tw}}
\maketitle % make the title area
%%
%%
\begin{flushleft}
  {\fontsize{12}{0} \bf 摘要}
\end{flushleft}
%
%
\begin{abstract}
  \hspace{2em}
  本研究旨在基於深度學習技術,改善印刷電路板(PCB)中線路的瑕疵檢測問題,以及提升在自動光學檢測(AOI)應用中的錯誤預警能力。本次研究中使用黑白PCB資料集,包含六種常見瑕疵:copper、good、mousebite、open、pin-hole、short及spur。另外為了解決實際應用中可能發生的瑕疵資料集數量不足問題,我們團隊提出透過生成對抗網絡(GAN)擴充資料集,並設置多種評估指標以確保生成影像的可靠性與實用性。
\end{abstract}
\begin{abstract}
  \hspace{2em}
  模型訓練部分,本研究比較多個常見的神經網絡模型,包括AlexNet、ConvNet、EfficientNet、GoogLeNet、MNASNet、MobileNet、ResNet、SqueezeNet、VGG、ViT及Xception,並調整學習率(learning rate)、訓練輪數(epoch)、批量大小(batch size)等參數,尋找最佳配置。對實驗中各模型的訓練結果進行觀察,結合過程中記錄的各種指標,進行綜觀的比較與分析,以進一步評估其在AOI應用中的實際效果。
\end{abstract}
%%
%%
\begin{IEEEkeywords}
    {\fontsize{10}{0} \bf 關鍵字:}深度學習、印刷電路板、自動光學檢測、瑕疵檢測、生成對抗網絡。
\end{IEEEkeywords}
%%
%%
\IEEEpeerreviewmaketitle
